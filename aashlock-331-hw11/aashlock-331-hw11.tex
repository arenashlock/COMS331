\documentclass[12pt]{article}
\usepackage{epsf}
\usepackage{amssymb}
\usepackage{enumitem}
\usepackage{amsmath}
\usepackage{tikz}
\usetikzlibrary{automata, positioning, arrows}

\title{aashlock-331-hw11}
\author{Aren Ashlock}
\date{April ??, 2024}

\setlength{\oddsidemargin}{-0.25in}
\setlength{\topmargin}{-0.5in}
\setlength{\headheight}{0cm}
\setlength{\headsep}{0cm}
\setlength{\textheight}{10in}
\setlength{\textwidth}{7in}
\setlength{\topskip}{0cm}

\begin{document}

\noindent\textbf{ComS 331 \quad Spring 2024 \quad Name: Aren Ashlock}

\begin{enumerate}

%----------------------------------- Q1 DONE -----------------------------------

\item Use reduction to prove that the language $L = \{\rho(M_1)\rho(M_2) : L(M_1) \cup L(M_2) = \Sigma^*\}$ is not decidable (you may assume that $M_1$ and $M_2$ have the same input alphabet $\Sigma$).

\color{blue}
    We can reduce the Halting Problem to $L$ by doing the following. Construct $M_1$ such that $L(M_1)$ accepts the input if $M$ halts on $w$ while $M_2$ is constructed such that $L(M_2)$ accepts all inputs.

    If $M$ halts on $w$, then $\rho(M_1)\rho(M_2) \in L$ since $L(M_1) \cup L(M_2) = \Sigma^*$. If $M$ doesn't halt, then $\rho(M_1)\rho(M_2) \notin L$ since $L(M_1) \cup L(M_2) \neq \Sigma^*$.

    This shows a potential way for $L$ to be decided. However, this is a contradiction since we can reduce the Halting Problem to $L$, which means $L$ is not decidable.
\color{black}

%-------------------------------------------------------------------------------

%----------------------------------- Q2 DONE -----------------------------------

\item Define an unrestricted grammar for the language $\{a^{n^2} : n \in \mathbb{N}\}$.

\color{blue}
    $G = (\{[, ], S, A, L, R, X\}, \{a\}, S, P)$

    $P = \{
    S \rightarrow [AL],\\
    aL \rightarrow La,\\
    AL \rightarrow LA,\\
    $$[L \rightarrow [R,\\
    L] \rightarrow X,\\
    Ra \rightarrow aR,\\
    RA \rightarrow aAR,\\
    R] \rightarrow AAL],\\
    aX \rightarrow Xa,\\
    AX \rightarrow Xa,\\
    $$[X \rightarrow \epsilon
    \}$
\color{black}

%-------------------------------------------------------------------------------

\end{enumerate}
\end{document}